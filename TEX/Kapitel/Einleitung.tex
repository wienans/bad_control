%!TEX root = ../main.tex
\chapter{Einleitung}

Autonom fahrende Fahrzeuge werden in Kombination mit der Elektromobilität heut\-zutage immer relevanter und viele Firmen arbeiten an Konzepten zum autonomen Fahren. Unter einem autonom fahrenden oder auch führerlosem Fahrzeug ver\-steht man allgemein ein Fahrzeug, welches ohne menschliches Eingreifen sicher am Straßenverkehr teilnehmen kann. Dies ist das 5. und höchste Level des autonomen Fahrens. Der Weg dorthin ist in weitere vier Level abgestuft und beschreibt Teilziele auf diesem Weg, wie zum Beispiel Spurhalteassistenten, Parkassistenten und vieles mehr.\\
Um Studierenden die Möglichkeit zu bieten während ihres Studiums bereits die Welt des autonomen Fahrens erkunden zu können, möchte das Unternehmen Bertrandt einen Modellbausatz für Hochschulgruppen entwickeln, damit diese sich in Wett-kämpfen in verschiedenen Disziplinen miteinander messen können.\\
In Zusammenarbeit mehrerer Thesen wird ein Prototyp entworfen, welcher später auch als Hardwarevorlage für die Hochschulgruppen dienen soll. In dieser Arbeit werde ich mich, aufgrund meines Interesses, für den Reglerentwurf und das autonome Fahren damit beschäftigen, eine geeignete Fahrzeugregelung zu entwickeln und programmiertechnisch umzusetzen. \\
Am Ende dieser Arbeit soll das Modellfahrzeug mit einer geeigneten Regelung ausgestattet sein, sodass es sicher einer Fahrspur folgen kann und auf eine Soll\-ge\-schwin\-dig\-keit regelt.\\
