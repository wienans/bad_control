%!TEX root = ../main.tex
\chapter{Stand der Technik}
\label{ch:Stand}
\subsection*{Das Straßenfahrzeug}
\label{ch:Stand_Strasse}
Heutige Fahrzeuge profitieren bereits von modernen Assistenzsystemen, wie Spur\-halte- und Spur\-wechsel\-assistenten, sowie einer adaptiven Geschwindigkeitsregelung (ACC). In hoch entwickelten Fahrzeugen können diese auch Straßenschilder erkennen und die Maximalgeschwindigkeit daran anpassen. Falls das Fahrzeug einem vorausfahrenden Fahrzeug zu nahe kommt, bremst es gegebenenfalls auf die Geschwindigkeit dieses Fahrzeuges ab.\\
Für die Detektierung der Umwelt sind die verbauten Sensoren essenziell und werden teilweise speziell für ein Fahrzeug entwickelt. Einer der meist verbauten Sensoren ist hierbei die Kamera. Sie ist im einfachsten Fall dafür zuständig, die Fahrspur zu erkennen und somit eine Basis für die Spurhalte- und Spurwechselassistenten zu bieten. Auf einem leistungsstarken Rechner werden dann aufwendige Bildverarbeitungsalgorithmen ausgeführt, um Daten wie Spurbreite, Abstand zur Spurmitte, Kurvenradius, gestrichelte oder durchgezogene Linien oder auch Spurnummer aus dem Kamerabild zu extrahieren. Diese Daten werden vorverarbeitet, damit der Reg\-ler des Spurhalteassistenten eine geeignete Eingangsgröße erhält, um das Fahrzeug in der Fahrbahnmitte zu halten. Die systemtechnische Auslegung des Reglers erfolgt durch eine Modellierung der Fahrzeugdynamik. Hierfür wird in der Regel das Einspurmodell von Riekert und Schunck verwendet, welches bereits 1940 zur Analyse des Lenkverhaltens bei Seitenwinden entwickelt wurde. \cite{Einspur} Für die Abstands- oder Geschwindigkeitsregelung werden zurzeit vorwiegend Radar- und Hallsensoren benutzt. Die gemessenen Werte, wie der Abstand zu den vorausfahrenden Fahrzeugen und die aktuelle Eigengeschwindigkeit, werden an das Regelsystem weiter\-ge\-lei\-tet. Mit diesen Daten kann das Regelsystem entweder die Geschwindigkeit oder den Abstand des Fahrzeuges zu vorherfahrenden Fahrzeugen regeln.\cite{WegAuto}
\subsection*{Das Modellfahrzeug}
Im Bereich der Modellfahrzeuge ist Audi mit ihren 1:8 Modellfahrzeugen für ihren 'Autonomous Driving Cup' ein Vorreiter in der Technik. Dort wird eine 3D-Kamera zur Erfassung des vorderen Umfeldes verwendet. Neben dem Bild, welches für die oben bereits erwähnte Spurerkennung benutzt wird, ist es auch möglich, Abstandsinformationen aus dem Bild zu errechnen. Diese werden allerdings nicht für eine Abstandsregelung benutzt, sondern für die Erkennung von Objekten auf und neben der Fahrbahn. Um die Abstandsregelung dennoch zu realisieren, ist das Modellfahrzeug mit Ultraschallsensoren ausgestattet, welche Abstände in verschiedenen Bereichen des Modellfahrzeuges messen. \cite{AUDI}
\subsection*{Verfahren zur Querregelung}
Die einfachste Möglichkeit einer Fahrspur zu folgen, ist eine Zweipunktregelung. Befindet sich das Fahrzeug rechts der Fahrbahnmitte, so wird nach links gelenkt und analog in die andere Richtung. Ohne weitere Betrachtung ist zu erkennen, dass dieses Regelverfahren nicht ausreichend ist, da die effektive Fahrspur einer Slalom fahrt um die Fahrbahnmitte ähnelt.\\
PD- und PID-Regler in unterschiedlichen Strukturen nehmen einen sehr großen Be\-reich in der Fahrzeugregelung ein. Sie sind flexibel und selbst ohne ein genaues mathe\-matisches Modell der Regelstrecke gut zu entwerfen. Toyota und Honda haben bereits Projekte mit PID-Reglern für die Querregelung mit guten Testergebnissen absolviert. \\
Eine weitere Methode ist die Zustandsregelung. Sie benötigt ein genaues mathematisches Modell, um die Parameter für die Regelung zu berechnen. Dieser Regler wird meistens im Zusammenhang mit dem Einspurmodell von Riekert und Schnuck benutzt. Gegenüber dem PID-Regler kann dieser Regler automatisch parametrisiert werden.\cite{FaDyRe}\cite{MAQuer}\\


