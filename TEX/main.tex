\documentclass{scrbook} %Dokumentenklasse Buch

%!TEX root = ../main.tex

% Set german to default language and load english as well
\usepackage[english,ngerman]{babel}

% Set UTF8 as input encoding
\usepackage[utf8]{inputenc}

% Set T1 as font encoding
\usepackage[T1]{fontenc}
% Load a slightly more modern font
\usepackage{lmodern}
% Use the symbol collection textcomp, which is needed by listings.
\usepackage{textcomp}
% Load a better font for monospace.
\usepackage{courier}

% Set some options regarding the document layout. See KOMA guide
\KOMAoptions{%
  paper=a4,
  fontsize=12pt,
  parskip=half,
  headings=normal,
  BCOR=1cm,
  headsepline,
  DIV=12}

% do not align bottom of pages
\raggedbottom

% set style of captions
\setcapindent{0pt} % do not indent second line of captions
\setkomafont{caption}{\normalsize} %(Bild)-Unterschriftsgröße (normalsize / small)
\setkomafont{captionlabel}{\bfseries} % Label für Fett bei Bildunterschrift
\setcapwidth[c]{0.9\textwidth}

% set the style of the bibliography
\bibliographystyle{alphadin}

% load extended tabulars used in the list of abbreviation
\usepackage{tabularx}

% load the color package and enable colored tables
\usepackage[table]{xcolor}

% define new environment for zebra tables
\newcommand{\mainrowcolors}{\rowcolors{1}{maincolor!25}{maincolor!5}}
\newenvironment{zebratabular}{\mainrowcolors\begin{tabular}}{\end{tabular}}
\newcommand{\setrownumber}[1]{\global\rownum#1\relax}
\newcommand{\headerrow}{\rowcolor{maincolor!50}\setrownumber1}

% add main color to section headers
\addtokomafont{chapter}{\color{maincolor}}
\addtokomafont{section}{\color{maincolor}}
\addtokomafont{subsection}{\color{maincolor}}
\addtokomafont{subsubsection}{\color{maincolor}}
\addtokomafont{paragraph}{\color{maincolor}}

% do not print numbers next to each formula
\usepackage{mathtools}
%\mathtoolsset{showonlyrefs} %Nummern an den Formeln ausschalten
% left align formulas
%\makeatletter
%\@fleqntrue\let\mathindent\@mathmargin \@mathmargin=\leftmargini
%\makeatother

% Allow page breaks in align environments
\allowdisplaybreaks

% header and footer
\usepackage{scrpage2}
\pagestyle{scrheadings}
\setkomafont{pagenumber}{\normalfont\sffamily\color{maincolor}}
\setkomafont{pageheadfoot}{\normalfont\sffamily}
\setheadsepline{0.5pt}[\color{maincolor}]

% German guillemets quotes
\usepackage[german=guillemets]{csquotes}

% load TikZ to draw diagrams
\usepackage{tikz}

% load additional libraries for TikZ
\usetikzlibrary{%
  automata,%
  positioning,%
}

% set some default options for TikZ -- in this case for automata
\tikzset{
  every state/.style={
    draw=maincolor,
    thick,
    fill=maincolor!18,
    minimum size=0pt
  }
}

% load listings package to typeset sourcecode
\usepackage{listings}

% set some options for the listings package
\lstset{%
  upquote=true,%
  showstringspaces=false,%
  captionpos=b,%
  basicstyle=\ttfamily,%
  keywordstyle=\color{keywordcolor}\slshape,%
  commentstyle=\color{commentcolor}\itshape,%
  stringstyle=\color{stringcolor}}
\renewcommand{\lstlistingname}{Quelltext}
\renewcommand{\lstlistlistingname}{Quelltextverzeichnis}

% enable german umlauts in listings
\lstset{
  literate={ö}{{\"o}}1
           {Ö}{{\"O}}1
           {ä}{{\"a}}1
           {Ä}{{\"A}}1
           {ü}{{\"u}}1
           {Ü}{{\"U}}1
           {ß}{{\ss}}1
}

% define style for pseudo code
\lstdefinestyle{pseudo}{language={},%
  basicstyle=\normalfont,%
  morecomment=[l]{//},%
  morekeywords={for,to,while,do,if,then,else},%
  mathescape=true,%
  columns=fullflexible}

% load the AMS math library to typeset formulas
\usepackage{amsmath}
\usepackage{amsthm}
\usepackage{thmtools}
\usepackage{amssymb}

% load the paralist library to use compactitem and compactenum environment
\usepackage{paralist}

% load varioref and hyperref to create nicer references using vref
\usepackage[ngerman]{varioref}
\PassOptionsToPackage{hyphens}{url} % allow line break at hyphens in URLs
\usepackage{hyperref}

% setup hyperref
\hypersetup{breaklinks=true,
            pdfborder={0 0 0},
            ngerman,
            pdfhighlight={/N},
            pdfdisplaydoctitle=true}

% Fix bugs in some package, e.g. listings and hyperref
\usepackage{scrhack}

% Allow todos
\usepackage{todonotes}

% define german names for referenced elements
% (vref automatically inserts these names in front of the references)
\labelformat{figure}{Abbildung\ #1}
\labelformat{table}{Tabelle\ #1}
\labelformat{appendix}{Anhang\ #1}
\labelformat{chapter}{Kapitel\ #1}
\labelformat{section}{Abschnitt\ #1}
\labelformat{subsection}{Unterabschnitt\ #1}
\labelformat{subsubsection}{Unterunterabschnitt\ #1}
\AtBeginDocument{\labelformat{lstlisting}{Quelltext\ #1}}

% define theorem environments
\declaretheorem[numberwithin=chapter,style=plain]{Theorem}
\labelformat{Theorem}{Theorem\ #1}

\declaretheorem[sibling=Theorem,style=plain]{Lemma}
\labelformat{Lemma}{Lemma\ #1}

\declaretheorem[sibling=Theorem,style=plain]{Korollar}
\labelformat{Korollar}{Korollar\ #1}

\declaretheorem[sibling=Theorem,style=definition]{Definition}
\labelformat{Definition}{Definition\ #1}

\declaretheorem[sibling=Theorem,style=definition]{Beispiel}
\labelformat{Beispiel}{Beispiel\ #1}

\declaretheorem[sibling=Theorem,style=definition]{Bemerkung}
\labelformat{Bemerkung}{Bemerkung\ #1}

%INCLUDE Packages
\usepackage{listings}
\lstdefinestyle{customc}{
  belowcaptionskip=1\baselineskip,
  breaklines=true,
  frame=single,
  xleftmargin=\parindent,
  language=C,
  showstringspaces=false,
  basicstyle=\footnotesize\ttfamily,
  keywordstyle=\bfseries\color{green!40!black},
  commentstyle=\itshape\color{purple!40!black},
  identifierstyle=\color{blue},
  stringstyle=\color{orange},
  numbers=left,                    % where to put the line-numbers; (none,left,right)
  numbersep=5pt,                   % how far the line-numbers are from the code
}
\lstset{escapechar=@,style=customc}
\usepackage{float}

\usepackage{nomencl}           % Symbolverzeichnis
\makenomenclature

% Layout-Einstellungen
\widowpenalty = 10000             % keine "Hurenkinder"
\clubpenalty = 10000              % keine "Schusterjungen"
\interfootnotelinepenalty=10000   % kein Seitenumbruch in Fussnoten % Style-Festlegen
%!TEX root = ../main.tex

% Use this file to define some macros you need in your thesis. A macro is a short command that inserts some mathematical symbols or texts you do not want to retype each time you need some. I recommend to use as many macros as possible, because you are able to change them later. For example if you use the same macro each time you need to give the formal semantics of an expression you can easily change the appearance of these brackets by updating the macro later on.

% Set of natural numbers
\newcommand{\N}{\mathbb{N}}

% The default epsilon does not look very nice
\let\epsilon\varepsilon

% If you need to use mathematical expressins like an epsilon in the section titles of your thesis you will end up with warnings that these special symbols cannot be included in the PDF favorites. The following macro uses the mathematical symbol during the text of the thesis and the string "Epsilon" in the PDF favorites.
\newcommand{\pdfepsilon}{\texorpdfstring{$\epsilon$}{Epsilon}}
 % Neue Kurzbefehle
%!TEX root = ../main.tex
%Regelungstechnik
\hyphenation{Reg-ler Reg-lers Reg-lung Re-ge-lung Führungs-sprung-antwort  Fre-quenz-gänge Reg-ler-aus-gangs-sig-nal Aus-le-ge-re-geln Zwei-punkt-reg-lung Ge-schwin-dig-keits-reg-ler}
%Technik/Mathematik
\hyphenation{Pa-ra-me-ter Va-ri-anzen Ma-the-ma-tik Ko-or-di-na-ten-system Ite-ra-tions-schritt ent-wi-ckelt Ent-wick-lung Mul-ti-pli-ka-tion}
%allgemein
\hyphenation{re-la-tiv al-ler-dings Fahr-bahn-mitte Pro-blem be-steht Ge-schwin-dig-keit Be-reich so-ge-nannte re-a-li-sie-ren dem-ent-spre-chend}
%Trennung
% PDF Titel und Autor setzen
\hypersetup{
  pdftitle={b.ad Regelung},
  pdfauthor={Sven Wienand}
}

%%!TEX root = ../main.tex

% define color of example university
\xdefinecolor{exampleuniversity}{rgb}{0.54,0.70, 0.22}

\colorlet{maincolor}{exampleuniversity}

\colorlet{stringcolor}{green!60!black}
\colorlet{commentcolor}{black!50}
\colorlet{keywordcolor}{maincolor!80!black}

\newcommand{\imagesuffix}{-color}
 %Farbschema Festlegen
%!TEX root = ../Bachelorarbeit.tex

\colorlet{maincolor}{black}

\colorlet{stringcolor}{black}
\colorlet{commentcolor}{black!50}
\colorlet{keywordcolor}{black}

\newcommand{\imagesuffix}{-gray}

\newcommand{\duedate}{31. Juli 2018}
\newcommand{\dueplace}{Bochum}
%Einzelne Seite zum Debugen includen
%\includeonly{einleitung}

\begin{document}
  \frontmatter %Römische Seitenzahlen-Einleitung
 	%!TEX root = ../main.tex

\begin{titlepage}
  \thispagestyle{empty}
  %\pgfimage[height=0.5cm]{./Bilder/Bertrandt_logo.png}
  

  \pgfimage[height=1cm]{./Bilder/logo\imagesuffix}
  \vskip2.5cm

  \LARGE
  
  \textbf{\sffamily\color{maincolor}b.ad Regelung}

  \textit{b.ad control}

  \normalfont\normalsize

  \vskip2em
  

  Autor: \\
  \textbf{\sffamily\color{maincolor}Sven Wienand}

  \vskip1em
 

  \vfill

  \dueplace, den \duedate
\end{titlepage}


  	%!TEX root = ../main.tex

\cleardoublepage
\thispagestyle{plain}

\pdfbookmark{Kurzfassung}{kurzfassung}
\paragraph{Kurzfassung}
In dieser Arbeit wird eine Fahrzeugregelung für ein spurgeführtes Modellfahrzeug entwickelt, welches mit einer kamerabasierten Fahrbahnerkennung und einer ultraschallbasierten Abstandsmessung zu potenziell vorher fahrenden Fahr\-zeu\-gen, parkenden Fahrzeugen oder Objekten ausgestattet ist. \\

Hierfür wird zu erst ein Überblick über die vorhandenen Hardware gegeben und die Systemarchitektur beschrieben. Im Anschluss daran werden die für die Regelung notwendigen regelungstechnischen und systemtechnischen Grundlagen angeführt und erklärt.\\

Die Querdynamik des Fahrzeuges wurde aufgrund fehlender Möglichkeiten, die Parameter des linearen Einspurmodells zu bestimmen, durch Testaufbauten \mbox{selbst} modelliert. Nach Auswahl des Regelverfahrens wurde ein PID-Regelalgorithmus erarbeitet und durch zwei Verfahren parametriert. Im Anschluss daran werden praktische Versuchsfahrten mit Simulationsergebnissen verglichen und analysiert.\\

Die Geschwindigkeitsregelung konnte wegen mechanischer Einschränkungen nur als Zweipunktregler ausgelegt werden. Die hierfür notwendige Ge\-schwindigkeits\-messung wurde mit einem Hallsensor realisiert. Als mögliche Erweiterung der Ge\-schwin\-dig\-keits\-messung wurde eine Sensor-Fusion von Beschleunigungs- und Ge\-schwin\-digkeits\-sensoren mithilfe eines Kalman-Filters erarbeitet und in Matlab simuliert.\\

Für eine zweite Version des Modellfahrzeuges wurde eine neue Systemarchitektur auf der Basis des Robot Operating System entwickelt. Der in dieser These er\-ar\-bei\-tete Regelalgorithmus wird in die ROS Umgebung portiert und ein Entwicklertest durchgeführt.

%Inhaltsverzeichnis
  \cleardoublepage %Zwingt auf neuer ungerader Seite anzufangen
  \phantomsection %hyperref zuordnungshilfe für PDF
  \pdfbookmark{Inhaltsverzeichnis}{tableofcontents}%PDF Link zur sektion
  \tableofcontents %Erzeuge Inhaltsverzeichnis
%Ende Inhaltsverzeichnis
  \mainmatter % Hautteil
  %!TEX root = ../main.tex
\chapter{Einleitung}

Autonom fahrende Fahrzeuge werden in Kombination mit der Elektromobilität heut\-zutage immer relevanter und viele Firmen arbeiten an Konzepten zum autonomen Fahren. Unter einem autonom fahrenden oder auch führerlosem Fahrzeug ver\-steht man allgemein ein Fahrzeug, welches ohne menschliches Eingreifen sicher am Straßenverkehr teilnehmen kann. Dies ist das 5. und höchste Level des autonomen Fahrens. Der Weg dorthin ist in weitere vier Level abgestuft und beschreibt Teilziele auf diesem Weg, wie zum Beispiel Spurhalteassistenten, Parkassistenten und vieles mehr.\\
Um Studierenden die Möglichkeit zu bieten während ihres Studiums bereits die Welt des autonomen Fahrens erkunden zu können, möchte das Unternehmen Bertrandt einen Modellbausatz für Hochschulgruppen entwickeln, damit diese sich in Wett-kämpfen in verschiedenen Disziplinen miteinander messen können.\\
In Zusammenarbeit mehrerer Thesen wird ein Prototyp entworfen, welcher später auch als Hardwarevorlage für die Hochschulgruppen dienen soll. In dieser Arbeit werde ich mich, aufgrund meines Interesses, für den Reglerentwurf und das autonome Fahren damit beschäftigen, eine geeignete Fahrzeugregelung zu entwickeln und programmiertechnisch umzusetzen. \\
Am Ende dieser Arbeit soll das Modellfahrzeug mit einer geeigneten Regelung ausgestattet sein, sodass es sicher einer Fahrspur folgen kann und auf eine Soll\-ge\-schwin\-dig\-keit regelt.\\

  %!TEX root = ../main.tex
\chapter{Stand der Technik}
\label{ch:Stand}
\subsection*{Das Straßenfahrzeug}
\label{ch:Stand_Strasse}
Heutige Fahrzeuge profitieren bereits von modernen Assistenzsystemen, wie Spur\-halte- und Spur\-wechsel\-assistenten, sowie einer adaptiven Geschwindigkeitsregelung (ACC). In hoch entwickelten Fahrzeugen können diese auch Straßenschilder erkennen und die Maximalgeschwindigkeit daran anpassen. Falls das Fahrzeug einem vorausfahrenden Fahrzeug zu nahe kommt, bremst es gegebenenfalls auf die Geschwindigkeit dieses Fahrzeuges ab.\\
Für die Detektierung der Umwelt sind die verbauten Sensoren essenziell und werden teilweise speziell für ein Fahrzeug entwickelt. Einer der meist verbauten Sensoren ist hierbei die Kamera. Sie ist im einfachsten Fall dafür zuständig, die Fahrspur zu erkennen und somit eine Basis für die Spurhalte- und Spurwechselassistenten zu bieten. Auf einem leistungsstarken Rechner werden dann aufwendige Bildverarbeitungsalgorithmen ausgeführt, um Daten wie Spurbreite, Abstand zur Spurmitte, Kurvenradius, gestrichelte oder durchgezogene Linien oder auch Spurnummer aus dem Kamerabild zu extrahieren. Diese Daten werden vorverarbeitet, damit der Reg\-ler des Spurhalteassistenten eine geeignete Eingangsgröße erhält, um das Fahrzeug in der Fahrbahnmitte zu halten. Die systemtechnische Auslegung des Reglers erfolgt durch eine Modellierung der Fahrzeugdynamik. Hierfür wird in der Regel das Einspurmodell von Riekert und Schunck verwendet, welches bereits 1940 zur Analyse des Lenkverhaltens bei Seitenwinden entwickelt wurde. \cite{Einspur} Für die Abstands- oder Geschwindigkeitsregelung werden zurzeit vorwiegend Radar- und Hallsensoren benutzt. Die gemessenen Werte, wie der Abstand zu den vorausfahrenden Fahrzeugen und die aktuelle Eigengeschwindigkeit, werden an das Regelsystem weiter\-ge\-lei\-tet. Mit diesen Daten kann das Regelsystem entweder die Geschwindigkeit oder den Abstand des Fahrzeuges zu vorherfahrenden Fahrzeugen regeln.\cite{WegAuto}
\subsection*{Das Modellfahrzeug}
Im Bereich der Modellfahrzeuge ist Audi mit ihren 1:8 Modellfahrzeugen für ihren 'Autonomous Driving Cup' ein Vorreiter in der Technik. Dort wird eine 3D-Kamera zur Erfassung des vorderen Umfeldes verwendet. Neben dem Bild, welches für die oben bereits erwähnte Spurerkennung benutzt wird, ist es auch möglich, Abstandsinformationen aus dem Bild zu errechnen. Diese werden allerdings nicht für eine Abstandsregelung benutzt, sondern für die Erkennung von Objekten auf und neben der Fahrbahn. Um die Abstandsregelung dennoch zu realisieren, ist das Modellfahrzeug mit Ultraschallsensoren ausgestattet, welche Abstände in verschiedenen Bereichen des Modellfahrzeuges messen. \cite{AUDI}
\subsection*{Verfahren zur Querregelung}
Die einfachste Möglichkeit einer Fahrspur zu folgen, ist eine Zweipunktregelung. Befindet sich das Fahrzeug rechts der Fahrbahnmitte, so wird nach links gelenkt und analog in die andere Richtung. Ohne weitere Betrachtung ist zu erkennen, dass dieses Regelverfahren nicht ausreichend ist, da die effektive Fahrspur einer Slalom fahrt um die Fahrbahnmitte ähnelt.\\
PD- und PID-Regler in unterschiedlichen Strukturen nehmen einen sehr großen Be\-reich in der Fahrzeugregelung ein. Sie sind flexibel und selbst ohne ein genaues mathe\-matisches Modell der Regelstrecke gut zu entwerfen. Toyota und Honda haben bereits Projekte mit PID-Reglern für die Querregelung mit guten Testergebnissen absolviert. \\
Eine weitere Methode ist die Zustandsregelung. Sie benötigt ein genaues mathematisches Modell, um die Parameter für die Regelung zu berechnen. Dieser Regler wird meistens im Zusammenhang mit dem Einspurmodell von Riekert und Schnuck benutzt. Gegenüber dem PID-Regler kann dieser Regler automatisch parametrisiert werden.\cite{FaDyRe}\cite{MAQuer}\\



  %!TEX root = ../main.tex
\chapter{Modellbildung}

\section{Lineares Einspurmodell}
  %!TEX root = ../main.tex
\chapter{Analyse des Modells}

\section{strukturelle Steuerbarkeit}
\section{strukturelle Beobachtbarkeit}
\section{Steuerbarkeit}
\section{Beobachtbarkeit}
  \cleardoublepage
  \phantomsection
  \addcontentsline{toc}{chapter}{Literaturverzeichnis}
  %\pdfbookmark{Literaturverzeichnis}{bibliography}
  \bibliography{./Literatur/Literatur} %BibTeX Compilieren
  \appendix % Beginn des Anhangs
  %\backmatter % Nach dem Anhang alle Verzeichnisse

  \cleardoublepage
  \phantomsection
  \addcontentsline{toc}{chapter}{Abbildungsverzeichnis}
  %\pdfbookmark{Abbildungsverzeichnis}{listoffigures}
  \listoffigures

  \cleardoublepage
  \phantomsection
  \addcontentsline{toc}{chapter}{Tabellenverzeichnis}
  %\pdfbookmark{Tabellenverzeichnis}{listoftables}
  \listoftables

  %\cleardoublepage
  %\phantomsection
  %\pdfbookmark{Definitions- und Theoremverzeichnis}{listoftheorems}
  %\renewcommand{\listtheoremname}{Definitions- und Theoremverzeichnis}
  %\listoftheorems[ignoreall,show={Lemma,Theorem,Korollar,Definition}]

  %\cleardoublepage
  %\phantomsection
  %\pdfbookmark{Quelltextverzeichnis}{listoflistings}
  %\lstlistoflistings

  %!TEX root = ../main.tex

\cleardoublepage
\phantomsection
\addcontentsline{toc}{chapter}{Abkürzungsverzeichnis}
%\pdfbookmark{Abkürzungsverzeichnis}{abbreviations}
\chapter*{Abkürzungsverzeichnis}
\label{section-abbrevs}
\begin{tabularx}{\textwidth}{lX}
  ACC & \textbf{A}ctive \textbf{C}ruise \textbf{C}ontrol, dt. Aktive Geschwindigkeitsregelung\\
  b.ad & \textbf{b}ertrandt \textbf{a}utonomous \textbf{d}riving\\
  SSH & \textbf{S}ecure \textbf{Sh}ell\\
  SCP & \textbf{S}ecure \textbf{C}o\textbf{p}y\\
  GCC & \textbf{G}NU \textbf{C}ompiler \textbf{C}ollection\\
  CPU & \textbf{C}entral \textbf{P}rocessing \textbf{U}nit, dt. (Haupt-) Prozessor\\
  GPU & \textbf{G}raphics \textbf{P}rocessing \textbf{U}nit, dt. Grafikprozessor\\
  RAM & \textbf{R}andom \textbf{A}ccess \textbf{M}emory, dt. Arbeitsspeicher\\
  OS & \textbf{O}perating \textbf{S}ystem, dt. Betriebssystem\\
  GPIO & \textbf{G}eneral \textbf{P}urpose \textbf{I}nput/\textbf{O}utput, dt. allgemeiner Ein- und Ausgang\\
  PWM & \textbf{P}ulse \textbf{W}idth \textbf{M}odulation , dt. Pulsweitenmodulation\\
  CSI & \textbf{C}amera \textbf{S}erial \textbf{I}nterface \\
  I$^2$C & \textbf{I}nter \textbf{I}ntegrated \textbf{C}ircuit \\  
  SPI & \textbf{S}erial \textbf{P}eripheral \textbf{I}nterface \\
  p & \textbf{P}ixel\\
  ROS & \textbf{R}obot \textbf{O}perating \textbf{S}ystem\\
\end{tabularx}

  

\end{document}
